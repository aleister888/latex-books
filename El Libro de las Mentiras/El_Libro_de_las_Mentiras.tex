% arara: xelatex

%%%%%%%%%%%%%%%%%%%%%%%%%%%%%%%
% Document class and packages %
%%%%%%%%%%%%%%%%%%%%%%%%%%%%%%%

\documentclass[8pt]{book}
\usepackage[utf8]{inputenc}
\usepackage[T1]{fontenc}
\usepackage{fontspec,titling,ragged2e,tocloft,changepage,xcolor}

%%%%%%%%%%
% Macros %
%%%%%%%%%%

% Macro: Indent multiple lines
\newenvironment{indent1}{\begin{adjustwidth}{12pt}{}}{\end{adjustwidth}}
\newenvironment{indent2}{\begin{adjustwidth}{24pt}{}}{\end{adjustwidth}}
\newenvironment{indent3}{\begin{adjustwidth}{36pt}{}}{\end{adjustwidth}}

% Add space to next line break
\newcommand{\spacinga}{\\\vspace{1pt}}
\newcommand{\spacingb}{\\\vspace{2.5pt}}
\newcommand{\spacingc}{\\\vspace{3.5pt}}

% Command for indenting line start
\newcommand{\aspacing}{\vspace{0.5pt}}
\newcommand{\bspacing}{\vspace{1.5pt}}
\newcommand{\cspacing}{\vspace{2.5pt}}

% Command for indenting next line
\newcommand{\indenta}{\\\hspace*{12pt}}
\newcommand{\indentb}{\\\hspace*{24pt}}
\newcommand{\indentc}{\\\hspace*{36pt}}

% Command for indenting line start
\newcommand{\nindenta}{\hspace*{12pt}}
\newcommand{\nindentb}{\hspace*{24pt}}
\newcommand{\nindentc}{\hspace*{36pt}}

% "?" and "!" boxes
\newcommand{\signbox}[1]{%
\thispagestyle{empty}

\vspace*{\fill}
\hspace{-25pt}
{\color{gray}
\centering
\setlength{\fboxrule}{5pt}
\framebox{\parbox{\textwidth}{\par\vspace{5.175cm}
\hspace{\fill}
\fontsize{50pt}{0pt}\selectfont \ #1
\hspace{\fill}
\par\vspace{5.175cm}}}}
\vspace*{\fill}
\newpage
}

% Sections
\newcommand{\seccion}[4]{%
%	Section Title
\section*{\centering#1}
\vspace{-8pt}
\section[KEΦAΛH #2]{\centering{\Large{KEΦAΛH #2}}}
\vspace{-8pt}
\subsection*{\normalfont\centering{#3}}
\vspace{2pt}
%	Section Text
\RaggedRight
#4
}

% Comments
\newcommand{\comment}[2]{
%	Comment Title
\subsection[\fontsize{8pt}{5pt}\selectfont COMENTARIO (#1)]{
\centering{\normalsize{\normalfont{
COMENTARIO (#1)
			}
		}
	}
}
%	Comment Text
\justifying
\vspace{-3pt}
\small{
\textit{
\hspace{4pt} #2
		}
	}
\vspace{-3pt}
}

% Subcomments
\newcommand{\subcomment}[1]{
\centering
\vspace{3pt}
\small{
\textit{
#1
		}
	}
}

% Comment footnotes
\newcommand{\commentfoot}[2]{
\begin{center}
{\textit{NOTA #1.- #2}}
\end{center}
}

%%%%%%%%%%%%%%%%%%%%%%%%%%%%%%
% Document Options and Looks %
%%%%%%%%%%%%%%%%%%%%%%%%%%%%%%

\setlength{\cftbeforetoctitleskip}{1em}

% Only print section numbers on the TOC
\setcounter{secnumdepth}{0}

% Change TOC title
\renewcommand{\contentsname}{Índice}

% Define page margins
\usepackage{geometry}
 \geometry{
 a6paper,
 left=12.5mm,
 right=12.5mm,
 top=7.5mm,
 bottom=15mm,
 }

% Set font
\setmainfont{Linux Libertine}

\usepackage[export]{adjustbox}

% Define page style
\usepackage[]{fancyhdr}
\pagestyle{fancy}
\fancyhead{}
\fancyfoot{}

% Set title, date and author
\title{\raggedright \fontsize{45pt}{40pt}\selectfont El Libro de las Mentiras}
\date{}
\author{}

%%%%%%%%%%%%%%%%%%
% Document Start %
%%%%%%%%%%%%%%%%%%

\begin{document}
% Makes title plage
\maketitle

\renewcommand{\headrulewidth}{0pt}

\signbox{?}

\signbox{!}

% Align text to left
\raggedright
\normalsize{Mentiras Título original: The Book of Lies: Which is also Falsely
Called BREAKS.The Wanderings or Falsifications of the One Thought of Frater
Perdurabo, which Thought is itself Untrue. Liber CCCXXXIII [Book 333]}

% TOC
\tableofcontents
% Define TOC Title
\addtocontents{toc}{\vspace{-1.25cm}}

\newpage

% Add page number to each page
\cfoot{\thepage}

\centering{%
\section[KEΦAΛH 0]{KEΦAΛH H OϒK EΣTI KEΦAΛH}
\vspace{-10pt}
\subsection*{¡O!\footnote[1]{%
Silencio, Nuit, O;Ra-Hoor-Khuit, Yo.}}

\small{
LA ANTEPRIMIGENIA TRÍADA QUE ES\\
EL NO-DIOS
\spacinga
Nada es.\\
Nada se hace.\\
Nada es no.
%
\spacingc
LA PRIMERA TRÍADA QUE ES EL DIOS
\spacinga
Yo revelo la Palabra.\\
Yo escucho la Palabra.
%
\spacingc
EL ABISMO
\spacinga
La palabra se ha roto\\
Sea el conocimiento.\\
El conocimiento es la Narración.\\
Estos fragmentos son Creación.\\
La fisura manifiesta la luz.\footnote[2]{%
Lo No Fragmentado, que todo lo integra, se llama Oscuridad.}
%
\spacingc
LA SEGUNDA TRÍADA QUE ES EL DIOS
\spacinga
DIOS Padre y Madre, oculta en la Generación.\\
DIOS oculto en la energía rotante de la Naturaleza.\\
DIOS manifestado en el hacemiento: armonía:\\
consideración: el espejo del Sol y del Corazón
%
\spacingc
LA TERCERA TRÍADA QUE ES EL DIOS
\spacinga
Paciencia: preparación.\\
Incertidumbre: fluyente, centelleante.\\
Estabilidad: procreación.
%
\spacingc
LA DÉCIMA EMANACIÓN
\spacinga
El Mundo.}}

\newpage

\comment{El Capitulo que no es un Capítulo}{
Este capítulo, numerado con el 0, se corresponde con lo Negativo, lo
que se halla antes de Kether en el sistema cabalístico.\indenta
La interrogación y la exclamación de las páginas anteriores son los
otros dos velos.\indenta
El significado de estos símbolos se explica en su totalidad en «El
Soldado y el Jorobado».\indenta
Este capítulo comienza con la letra O, seguida de una exclamación; su
remisión a la teogonía del «Libro de la Ley» se explica en una nota, pero
también se refiere a KTEIS PHALLOS y SPERMA; es la exclamación que
expresa la maravilla o el éxtasis, que son la naturaleza final de las cosas.
}

\comment{La Antiprimigenia Tríada}{
Esta es la Trinidad negativa; sus tres manifestaciones, son en última
instancia, idénticas. Armonizan el Ser, el Devenir y el No-Ser. los tres
modos de concebir el universo.\indenta
La manifestación Nada es No, Técnicamente equivalente a Algo es,
se explica ampliamente en el capítulo titulado Berashith.\indenta
El resto del capítulo sigue el sistema Sefirótico de la Cábala, y
constituye una suerte de comentario quintaesenciado de dicho sistema.\indenta
Aquellos que estén familiarizados con el sistema reconocerán a
Kether, Chokmah y Binah en la Primera Tríada; a Daath en el Abismo;
a Chesed, Geburah y Tiphareth, en la Segunda Tríada; a Netzach, Hod y
Yesod en la Tercera Tríada y a Malkuth en la Décima Emanación.\indenta
Se advertirá que esta cosmogonía es muy completa; incluso la
manifestación de Dios no aparece hasta Tiphareth; y el universo mismo
hasta Malkuth.
}

\vspace{-1.25pt}

\subcomment{
El capítulo puede considerarse, por lo tanto, como el tratado sobre la
existencia más completo que se haya escrito.
}

\newpage

\seccion{1}{A}
{EL SABBATH DEL CHIVO}{
¡O!, el corazón de N.O.X, la Noche de Pan.\\
ΠAN: Dualidad: Energía: Muerte.\\
Muerte: Procreación: los valedores de ¡O!\\
Procrear es morir: morir es procrear.\\
Esparce la Semilla en el Campo de la Noche.\\
La Vida y la Muerte son dos nombres de A.\\
Mátame\\
Ninguna de estas cosas solas es suficiente.
}

\newpage

\comment{A}{
La forma del 1 sugiere el Falo; este capítulo se titula, así pues, El
Sabbath del Chivo, el Sabbath de las Brujas, donde se adora el Falo.\indenta
El capítulo comienza con la repetición de ¡O!, que remite al capítulo
anterior. Se explica que esta tríada vive en la Noche, la Noche del Pan,
llamada místicamente N.O.X., y esta O es la O de la palabra. La N es un
símbolo del Tarot, la Muerte; y la X o Cruz es el signo del Falo. Para un
comentario más exhaustivo sobre Nox, véase Liber VII, capítulo I.\indenta
Nox se agrega al 210, que simboliza la reducción de la dualidad a
unidad, y de ahí la negatividad, y es, por lo tanto, un jeroglífico de la
Gran Labor.\indenta
La palabra Pan se explica, Π, la letra de Marte, es un jeroglífico de dos
pilares, que sugiere pues la dualidad; la A, por su forma, es el
pentagrama, la energía; y la N, por sus atributos en el Tarot, es la
muerte.\indenta
Nox queda explicada, y ello muestra que la última Trinidad ¡O!, se
sostiene, o alimenta, por el proceso de muerte y procreación, las leyes del
universo.\indenta
La identidad de ambas queda explicada.\indenta
El estudiante tiene en cargo la responsabilidad de comprender la
importancia espiritual de este proceso físico en la línea 5.\indenta
Se acepta, pues, que la última letra A tiene dos nombres, o fases, Vida
y Muerte.\indenta
La línea 7 equilibra a la línea 5. Se advertirá que la fraseología de
estas dos líneas es tan meditada que una contiene a la otra más que a sí
misma.
}

\subcomment{La línea 8 enfatiza la importancia de activar ambas}

\newpage

\seccion{2}{B}
{EL GRITO DEL HALCÓN}{
Hoor tenía un nombre secreto de cuatro palabras:\indenta
Haz lo que quieras\footnotemark[3].\\
Cuatro Palabras: Cero-Uno-Mucho-Todo.
\begin{indent2}
¡Tú, Niño!\\
Tu nombre es sagrado.\\
Tu Reino adviene.\\
Tu Voluntad se cumple.\\
Aquí está el Pan\\
Aquí está la Sangre.
\end{indent2}
\nindenta
¡Dánoslo mediante la tentación!\indenta
¡Líbranos del Bien y del Mal!\\
Sea tanto lo Tuyo como lo Mío la Corona del Reino,\indenta
Incluso ahora.\indentc
ABRAHADABRA\\
Estas diez palabras son cuatro, el Nombre del Uno.
}

\footnotetext[3]{Catorce letras. \textit{Quid Voles Illud Fac. Q.V.I.F. 196 = 14².}}

\newpage

\comment{B}{
El Halcón mencionado es Horus.\indenta
El capítulo comienza con un comentario del Libro de la Ley, III, 49.\indenta
Aquellas cuatro palabras, Haz Lo Que Quieras, se identifican\indenta
también con los cuatro modos posibles de concebir el universo: Horus los
une.\indenta
Sigue una versión del Padrenuestro, adecuada a Horus. Compárese
con la versión del capítulo 44. Hay diez partes en esta oración, y, como
oración atribuida a Horus, son cuatro, como se ha explicado; pero es solo
el nombre de Horus en cuatro palabras; Él mismo es Uno.\indenta
Puede compararse esto con la doctrina cabalística de los diez Sefirot,
como expresión del Tetragrámaton (1 más 2 más 3 más 4 = 10).\indenta
Se puede entender ahora que este Halcón no es solar sino mercurial;
de ahí las palabras, El Grito del Halcón, la parte esencial de Mercurio es
su Voz; y el número del capítulo, B, que es Beth, la letra de Mercurio, el
Mago del Tarot que tiene cuatro armas. Debe recordarse que esta carta
es la número 1, de nuevo en conexión todos estos símbolos con el Falo.
}

\subcomment{El arma esencial de Mercurio es Caduceo.}

\newpage

\seccion{3}{Γ}
{LA OSTRA}{
Los Hermanos de A.·.A.·. son uno con la Madre del Niño\footnotemark[4].\\
La Multitud es tan adorable para el Uno como el Uno lo es
\begin{indent1}
para la Multitud. He aquí el Amor de estos; el parto-creación\\
es la Gloria del Uno; el coito-disolución es la Gloria\\
de la Multitud.
\end{indent1}
El Todo, así tramado de estos, es la Gloria.\\
La Nada está más allá de la Gloria.\\
El Hombre se deleita en la unión con la Mujer; la Mujer,\indenta
en desprenderse del niño.\\
Los Hermanos de A.·.A.·. son Mujeres; los aspirantes a\\
A.·.A.·., hombres.
}

\newpage

\comment{Γ}{
Gimel es la Gran Sacerdotisa del Tarot. Este capítulo ofrece
un punto de vista de las féminas iniciadas; se titula, por lo
tanto, la Ostra, un simbolo del Yoni. En el {\normalfont{Equinoccio X}},
el Templo del Rey Salomón, se explica cómo los maestros del Templo,
o Hermanos de A.·. A.·. han variado la fórmula de sus progresos. Estas
dos {\normalfont{formulae}}, Solve et Coagula, se explican aquí, y el
universo se presenta como la interacción entre ellas dos. Esto también
explica la manifestación del {\normalfont{Libro de la Ley, I, 28-30}}.
}

\commentfoot{4}{Ellos hacen que todos los hombres lo adoren.
}

\newpage

\seccion{3}{Δ}
{MELOCOTONES}{
¡Suave y vacío, cómo conquistaste lo duro y lo lleno!\\
¡Muere, se entrega, para Ti es el fruto!\\
Sé tú la Esposa; tú serás la Madre de ahora en adelante.\\
Para todas las impreisones. No permitas que te conquisten;\\
permite que crezcan en ti. La mínima de las impresiones,\indenta
que alcanza su perfección, es Pan.\indenta
Recibe a un millar de amantes; y portarás solo un Niño.\\
Este niño será el heredero del Destino del Padre.
}

\end{document}
