% arara: xelatex
\documentclass[12pt,b6paper]{book}
\usepackage[utf8]{inputenc}
\usepackage[T1]{fontenc}
\usepackage{fontspec,titling,ragged2e,tocloft}

\setlength{\cftbeforetoctitleskip}{1em}

% Only print section numbers on the TOC
\setcounter{secnumdepth}{0}

% Change TOC title
\renewcommand{\contentsname}{Índice}

% Macro: Sections
%
\newcommand{\seccion}[4]{%
% Section Title
\section*{\centering#1}
\vspace{-8pt}
\section[KEΦAΛH #2]{\centering{\Large{KEΦAΛH #2}}}
\vspace{-8pt}
\subsection*{\normalfont\centering{#3}}
\vspace{2pt}
% Section Text
	\begin{center}
		\begin{tabular}{l}
	#4
		\end{tabular}
	\end{center}
}



% Macro: Comments
%
\newcommand{\comment}[2]{
% Comment Title
\subsection[\small COMENTARIO (#1)]{
	\centering{\normalsize{\normalfont{
		COMENTARIO (#1)
}}}
}
% Comment Text
\justifying
\small{
	\textit{
\hspace{4pt} #2
		}
	}
}

\newcommand{\subcomment}[1]{
\centering
\small{
	\textit{
#1
	}
}
% command
}

% Add space to next line break
\newcommand{\spacing}{\\\vspace{1.5pt}}
\newcommand{\spacingg}{\\\vspace{3pt}}
\newcommand{\spacinggg}{\\\vspace{9pt}}

% Command for indenting line start
\newcommand{\sspacing}{\vspace{1.5pt}}
\newcommand{\ssspacing}{\vspace{3pt}}
\newcommand{\sssspacing}{\vspace{9pt}}

% Command for indenting next line
\newcommand{\indentt}{\\\hspace*{12pt}}
\newcommand{\indenttt}{\\\hspace*{24pt}}
\newcommand{\indentttt}{\\\hspace*{36pt}}

% Command for indenting line start
\newcommand{\nindentt}{\hspace*{12pt}}
\newcommand{\nindenttt}{\hspace*{24pt}}
\newcommand{\nindentttt}{\hspace*{36pt}}

% Define page margins
\usepackage{geometry}
 \geometry{
 a5paper,
 left=20mm,
 right=20mm,
 top=15mm,
 }

% Set font
\setmainfont{Linux Libertine}

\usepackage[export]{adjustbox}

% Define page style
\usepackage[]{fancyhdr}
\pagestyle{fancy}
\fancyhead{}
\fancyfoot{}

% Set title, date and author
\title{\raggedright \fontsize{60pt}{50pt}\selectfont El Libro de las Mentiras}
\date{}
\author{}

% Begin document
\begin{document}
\pagenumbering{gobble}
\maketitle
\newpage
\renewcommand{\headrulewidth}{0pt}
\pagenumbering{arabic}

\raggedright
\normalsize{Mentiras Título original: The Book of Lies: Which is also Falsely Called BREAKS. The Wanderings or Falsifications of the One Thought of Frater Perdurabo, which Thought is itself Untrue. Liber CCCXXXIII [Book 333]}
 info page

\tableofcontents
\addtocontents{toc}{\vspace{-1.25cm}}

\newpage

\cfoot{\thepage}

\centering{%
\section[KEΦAΛH 0]{KEΦAΛH H OϒK EΣTI KEΦAΛH}
\vspace{-10pt}
\subsection*{¡O!\footnote[1]{%
Silencio, Nuit, O;Ra-Hoor-Khuit, Yo.}}

\small{
LA ANTEPRIMIGENIA TRÍADA QUE ES\\
EL NO-DIOS
\spacingg
Nada es.\\
Nada se hace.\\
Nada es no.
\spacinggg
LA PRIMERA TRÍADA QUE ES EL DIOS
\spacingg
Yo revelo la Palabra.\\
Yo escucho la Palabra.
\spacinggg
EL ABISMO
\spacingg
La palabra se ha roto\\
Sea el conocimiento.\\
El conocimiento es la Narración.\\
Estos fragmentos son Creación.\\
La fisura manifiesta la luz.\footnote[2]{%
Lo No Fragmentado, que todo lo integra, se llama Oscuridad.}
\spacinggg
LA SEGUNDA TRÍADA QUE ES EL DIOS
\spacingg
DIOS Padre y Madre, oculta en la Generación.\\
DIOS oculto en la energía rotante de la Naturaleza.\\
DIOS manifestado en el hacemiento: armonía:\\
consideración: el espejo del Sol y del Corazón
\spacinggg
LA TERCERA TRÍADA QUE ES EL DIOS
\spacingg
Paciencia: preparación.\\
Incertidumbre: fluyente, centelleante.\\
Estabilidad: procreación.
\spacinggg
LA DÉCIMA EMANACIÓN
\spacingg
El Mundo.}}

\newpage

\comment{El Capitulo que no es un Capítulo}{
Este capítulo, numerado con el 0, se corresponde con lo Negativo, lo
que se halla antes de Kether en el sistema cabalístico.\indentt
La interrogación y la exclamación de las páginas anteriores son los
otros dos velos.\indentt
El significado de estos símbolos se explica en su totalidad en «El
Soldado y el Jorobado».\indentt
Este capítulo comienza con la letra O, seguida de una exclamación; su
remisión a la teogonía del «Libro de la Ley» se explica en una nota, pero
también se refiere a KTEIS PHALLOS y SPERMA; es la exclamación que
expresa la maravilla o el éxtasis, que son la naturaleza final de las cosas.
}

\sssspacing

\comment{La Antiprimigenia Tríada}{
Esta es la Trinidad negativa; sus tres manifestaciones, son en última
instancia, idénticas. Armonizan el Ser, el Devenir y el No-Ser. los tres
modos de concebir el universo.\indentt
La manifestación Nada es No, Técnicamente equivalente a Algo es,
se explica ampliamente en el capítulo titulado Berashith.\indentt
El resto del capítulo sigue el sistema Sefirótico de la Cábala, y
constituye una suerte de comentario quintaesenciado de dicho sistema.\indentt
Aquellos que estén familiarizados con el sistema reconocerán a
Kether, Chokmah y Binah en la Primera Tríada; a Daath en el Abismo;
a Chesed, Geburah y Tiphareth, en la Segunda Tríada; a Netzach, Hod y
Yesod en la Tercera Tríada y a Malkuth en la Décima Emanación.\indentt
Se advertirá que esta cosmogonía es muy completa; incluso la
manifestación de Dios no aparece hasta Tiphareth; y el universo mismo
hasta Malkuth.
}

\subcomment{
El capítulo puede considerarse, por lo tanto, como el tratado sobre la
existencia más completo que se haya escrito.
}

\newpage

\seccion{1}{A}
{EL SABBATH DEL CHIVO}{
¡O!, el corazón de N.O.X, la Noche de Pan.\\
ΠAN: Dualidad: Energía: Muerte.\\
Muerte: Procreación: los valedores de ¡O!\\
Procrear es morir: morir es procrear.\\
Esparce la Semilla en el Campo de la Noche.\\
La Vida y la Muerte son dos nombres de A.\\
Mátame\\
Ninguna de estas cosas solas es suficiente.
}

\newpage

\comment{A}{
La forma del 1 sugiere el Falo; este capítulo se titula, así pues, El
Sabbath del Chivo, el Sabbath de las Brujas, donde se adora el Falo.\indentt
El capítulo comienza con la repetición de ¡O!, que remite al capítulo
anterior. Se explica que esta tríada vive en la Noche, la Noche del Pan,
llamada místicamente N.O.X., y esta O es la O de la palabra. La N es un
símbolo del Tarot, la Muerte; y la X o Cruz es el signo del Falo. Para un
comentario más exhaustivo sobre Nox, véase Liber VII, capítulo I.\indentt
Nox se agrega al 210, que simboliza la reducción de la dualidad a
unidad, y de ahí la negatividad, y es, por lo tanto, un jeroglífico de la
Gran Labor.\indentt
La palabra Pan se explica, Π, la letra de Marte, es un jeroglífico de dos
pilares, que sugiere pues la dualidad; la A, por su forma, es el
pentagrama, la energía; y la N, por sus atributos en el Tarot, es la
muerte.\indentt
Nox queda explicada, y ello muestra que la última Trinidad ¡O!, se
sostiene, o alimenta, por el proceso de muerte y procreación, las leyes del
universo.\indentt
La identidad de ambas queda explicada.\indentt
El estudiante tiene en cargo la responsabilidad de comprender la
importancia espiritual de este proceso físico en la línea 5.\indentt
Se acepta, pues, que la última letra A tiene dos nombres, o fases, Vida
y Muerte.\indentt
La línea 7 equilibra a la línea 5. Se advertirá que la fraseología de
estas dos líneas es tan meditada que una contiene a la otra más que a sí
misma.
}

\subcomment{La línea 8 enfatiza la importancia de activar ambas}

\newpage

\seccion{1}{B}
{EL GRITO DEL HALCÓN}{
Hoor tenía un nombre secreto de cuatro palabras:\indentt
Haz lo que quieras\footnotemark[3].\\
Cuatro Palabras: Cero-Uno-Mucho-Todo.\indenttt
¡Tú, Niño!\indenttt
Tu nombre es sagrado.\indenttt
Tu Reino adviene.\indenttt
Tu Voluntad se cumple.\indenttt
Aquí está el Pan\indenttt
Aquí está la Sangre.\indentt
¡Dánoslo mediante la tentación!\indentt
¡Líbranos del Bien y del Mal!\\
Sea tanto lo Tuyo como lo Mío la Corona del Reino,\indentt
Incluso ahora.\indentttt
ABRAHADABRA\\
Estas diez palabras son cuatro, el Nombre del Uno.
}

\footnotetext[3]{Catorce letras. \textit{Quid Voles Illud Fac. Q.V.I.F. 196 = 14².}}

\newpage

\comment{B}{
El Halcón mencionado es Horus.\indentt
El capítulo comienza con un comentario del Libro de la Ley, III, 49.\indentt
Aquellas cuatro palabras, Haz Lo Que Quieras, se identifican\indentt
también con los cuatro modos posibles de concebir el universo: Horus los
une.\indentt
Sigue una versión del Padrenuestro, adecuada a Horus. Compárese
con la versión del capítulo 44. Hay diez partes en esta oración, y, como
oración atribuida a Horus, son cuatro, como se ha explicado; pero es solo
el nombre de Horus en cuatro palabras; Él mismo es Uno.\indentt
Puede compararse esto con la doctrina cabalística de los diez Sefirot,
como expresión del Tetragrámaton (1 más 2 más 3 más 4 = 10).\indentt
Se puede entender ahora que este Halcón no es solar sino mercurial;
de ahí las palabras, El Grito del Halcón, la parte esencial de Mercurio es
su Voz; y el número del capítulo, B, que es Beth, la letra de Mercurio, el
Mago del Tarot que tiene cuatro armas. Debe recordarse que esta carta
es la número 1, de nuevo en conexión todos estos símbolos con el Falo.
}

\subcomment{El arma esencial de Mercurio es Caduceo.}

\end{document}
